\section{Discussion}

\subsection{Achievements}
The achievements of the project could be discussed on several aspects including function realization and the system stability.

After the main function realization completed, the features of the upgraded system could be listed as following:

\begin{itemize}
\item Login and register validation via jdbc
\item Secure the content within pairwise chat and group chat
\item Save encrypted signal storages and history messages at client local
\item Multi-device system
\item Message backup between devices
\item Work in both asynchronous and synchronous environment.
\item Fingerprint verification for pairwise chat.
\end{itemize}

In consideration of the system stability, the project uses junit testing and black-box testing. The junit testing covers almost all the logic functions that could be tested and all the junit tests have been passed. In black-box testing, different testing situations are considered. The issues discovered during the testing have been fixed at the end.

As a security project, the secure E2EE chat system achieves the goal of implementing Signal Protocol. After testing, the system could work as expected to secure the content during communication. The users' privacy within the system have been guaranteed basically. However, as a commercial product, the security of the whole system still needs to be improved like preventing SQL injection attack.

\subsection{Defects}
The final product of the project could not compete with the existed related commercial product due to the defects. The defects would be introduced with two parts including the client and server.

\begin{enumerate}[label=(\roman*)]
\item Defects in the client

In the Signal related aspect, the user could not verify the security of the group chat via fingerprint. The fingerprint of the pairwise chat could be generated and compared to check whether the session is monitored by adversary. The users could only perceive something wrong in group chat when the related pairwise chat has the failed verification. The break-in recovery feature is also not implemented in the client. In Signal system, the signed pre key is a mid-term key of a user. The signed pre key is required to be updated within several weeks for security consideration. Also the function that appends the users' pre keys is not implemented in the client. Although these functions are not essential for a secure E2EE chat system, but the missing of them could make the whole system vulnerable.

In the structure aspect, due to the existing project structure like IO structure, the system may occur unexpected errors sometimes. For example, when a user switches devices several times continuously, the group chat would not be reinitialized correctly. Since this project belongs to a security project, refactoring is not the own job during developing. But the issues caused by the project structure decreases the robustness of the whole system.

\item Defects in the server

In the Signal related aspect, the server does not implement the function that requesting new pre keys once the users' pre keys are used out. Although it does not influence the system working correctly, this would reduce the security of X3DH algorithm. Besides, the management of group chat is supposed to be enriched including adding or removing members.

In the project structure aspect, the persistence of undeliverable messages are not implemented. The server does not store them in database, once the server is rebooted, all the undeliverable messages would be lost. This defect would influence the system working incorrectly in asynchronous environment. As a secure chat system, it's a regret that the sever does not prevent other type's attacks such as SQL injection attack and DDoS attack.

In consideration of the performance, the login and registration verification in the server are also required to be optimized. The SQL execution times during the login verification is too much so that causing a large amount of time loss in the poor network environment. Besides due to the limited time and the multi threads structure, the sever may break down if numerous users login in the meanwhile.

\end{enumerate}

\subsection{Evolution}
Corresponding to the defects of the system, the evolution could be discussed on the client and server side too.

\begin{enumerate}[label=(\roman*)]
\item Evolution in the client

The fingerprint verification function of group chat could be implemented by using members' identity keys. The fingerprint of the group chat would be generated by combining all the members' identity key and members could compare it with each other to make sure the group chat is not monitored.

To achieve the break-in recovery feature, the users should be allowed to update their identity key and signed pre key. Once the user updated key bundle, all the other users would be informed and reinitialize the related chat immediately.

In the future development, the system could refactor the client structure including IO structure and connectionData format. Refactoring IO structure allows the client could handle the asynchronous functions more easier. The unexpected issues would decrease after the refactoring. The transmission data's format is also supposed to be refactored, all the payload could be serialized to bytes for decreasing the number of types in connectionData. Besides, it's better to refactor the transmission protocol to make the client could communicate with other Signal compatible sever. This could proof the Signal Protocol is implemented and working correctly inside the system.

\item Evolution in the sever

To improve the X3DH security, the sever is supposed to request new pre keys from users once the pre keys are used out. In the current system, the sever would return a null value back if there is no pre key related to the user.

As for the management of group chat, the sever does not verify if the sender is one of the members when the user doing some group operations. For example, if the user is out of the group, the user should not be allowed to ask all the other members to update the sender key. This alteration could provide the stability of group chat function.

While verify the user identifier during login and registration, the sever could combine several SQL executions together to reduce the time loss. Another solution of this performance problem is loading all the users information once the sever boots, then verifying user identifier in cache is faster then inquiry from database. But in this solution, the space would cost a lot if there are a large mount of registered users.

\end{enumerate}

\clearpage