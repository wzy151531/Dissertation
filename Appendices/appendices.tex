\section{Appendices}
The git repository address of this project is \href{https://git-teaching.cs.bham.ac.uk/mod-msc-proj-2019/zxw989}{https://git-teaching.cs.bham.ac.uk/mod-msc-proj-2019/zxw989}
\subsection{Project structure of git repository}
The structure of the git repository could be divided into two parts including socotra-client and socotra-server.

Both the client and the server project use gradle to manage. The logic code and testing code are contained in the /src/main and the /src/test directory respectively. Inside the /src/main directory, the functional part is developed in /java directory. The resources directory maintains the static files including the view files and property files for project config.

The main entry of the program in the server is the Server.java file. The core functions in the /java directory on the server side is developed inside several packages including common, jdbc and service. The common package are responsible for maintaining the common class both on client and server. The version of the common classes must be the same while development. The classes of jdbc package are used to handle the jdbc related operations. In the service package, there are classes for processing the communication with the clients.

The main entry of the program in the client is the Client.java file. The common package's duty is the same as it in the server. Since the architecture of the client project is MVC, the controller package holds the functions that control the view's changing operations and the model packages maintains the data model of the view. The implemented Signal related functions are developed in protocol package including Signal stores, encryption handler and file handler.

\subsection{Quick start of the code}
\subsubsection{Build and run}
To make the whole system work properly, the server project is required to run first. Before running the server project, it is essential to add a jdbc.properties file which includes the ssh connection information and database information in src/main/resources like:

\begin{lstlisting}
sshUser=bhamUsername
sshPassword=bhamPassword
dbUser=username
dbPassword=password
\end{lstlisting}

Because the database is severed by bham, for connecting to the bham's database service successfully, the server project is required to use SSH connection. The SSH information including sshUser and sshPassword which are them same as bham's identifier information. If the further development requires the db identifier information, please contact me with the email (grimwangziyan@163.com).

The two projects both use gradle-wrapper, developers could just open either of them in intellij or eclipse simply, and use the gradle tool inside the IDE to build and run the project.

The project could be simply run without the installing the gradle, just run the following commands under the project root folder like /socotra-server or /socotra-client:

\begin{lstlisting}
./gradlew build
\end{lstlisting}

then to run the application:

\begin{lstlisting}
./gradlew run
\end{lstlisting}

Or using the gradle GUI tool inside the IDE (for intellij it's on the right side bar) is more convenience to run the corresponding tasks.

\subsubsection{Junit test}

Both two projects use junit5 to write some tests exclude socket and GUI tests. To run the test, use the gradle tool test in menu Tasks/verification or use the command in the terminal:

\begin{lstlisting}
./gradlew test
\end{lstlisting}

The test report will automatically generated by gradle in directory build/reports/tests/test/index.html.

\subsubsection{Archive and release}

For server project, you can release the .jar file, you can archive the server project as:

\begin{lstlisting}
./gradlew uploadArchives
\end{lstlisting}

then the repos foler will be generated with serveral .jar files.
For client project, it use javafx11 to build the GUI, so once you use the gradle to generate a .jar file, you need to download javafx11 sdk first, and run it as following:

\begin{lstlisting}
java --module-path $PATHTOJAVAFXSDK11 --add-modules javafx.controls,javafx.fxml,javafx.base -jar $YOURCLIENT.jar
\end{lstlisting}

\clearpage