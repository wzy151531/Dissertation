\section{Overview}
This paper is about upgrading a chat system using Signal Protocol to implement a secure end-to-end-encryption chat system. This report focus on having a brief introduction about Signal Protocol and describing how to implementing a Signal Protocol based E2EE system including on client, server and database sides.
The existing chat system is developed by myself previously using gradle6 and jdk11. To improve the security of the communication, the project using Signal Protocol to encrypt the messages including on client, server and database sides.

\subsection{Abstract}
As people’s awareness of safety increases, normal symmetric encryption and asymmetric encryption like TLS, AES cannot satisfy their needs: even though the security of transmission can be guaranteed, the server still needs to decrypt the encrypted messages. So once the malicious people use it to get users' sensitive information on server side, users' privacy will be leaked. Besides, the normal symmetric and asymmetric encryptions keep  using the same cipher key during the transmission, once the hacker cracks one encrypted package, the previous and future encrypted packages will be exposed immediately.

In this situation, the Signal Protocol is developed to solve this problem. It implement an E2EE protocol to make sure only the call parties can decrypt the encrypted messages, and the cipher key will be changed in each transmission round which can not be referred forward or backward. The server does not decrypt the messages but only forwards the packages to the right users and the database also does not store the decrypted messages from users which greatly reduce the risk of user privacy leakage.

This report describes the fundamental of Signal Protocol and the whole process of implementing it on a existing chat system including on client, server and database sides. On client side, users' communication messages are encrypted by Signal Protocol, only the receiver and other essential information are exposed to the outside. On server side, the server handle the received packages corresponds to their type and only forward the encrypted messages to the corresponding users. Server does not use the libsignal-protocol-java packages which can guarantee the server cannot decrypt the messages. On database side, it only stores users' public keys for other users initializing the pairwise chat, so once the database is breached, there is no damage caused to users.

At the end, the project implements an end-to-end-encryption chat system including pairwise chat, group chat, local history messages storage, multi-device system and message backup functions etc.

Keywords: Signal Protocol, end-to-end-encryption, chat system

\subsection{Acknowledgements}
The success of the project cannot be separated from the help of many people and papers.

First I need to appreciate my Project Supervisor, Mihai Ordean, we have a detailed meeting every week. After each meeting, I can find new directions and questions. Also it's his progress to let me set this subject which I have never touched before.

I also need to thanks my friends, who have given me a lot of advice about this project and listened to my new ideas and my own explains to the project.

Finally, I need to appreciate my family, they spend endless hours to support my project by helping me find the related papers and test the project's robustness which is significant to the development of the project.

\clearpage