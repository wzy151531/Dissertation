\section{Conclusion}

\subsection{summary}
This project is aim to implement a secure E2EE chat system based on Signal Protocol. The final artifact achieves the goal set at the beginning. The previous product is a traditional chat system that both the client and the server could access the content of communication. Although TLS is used to secure the transmission package, some security features such as the forward security and future security are missing in the previous system. Once the one of the encrypted packages is cracked, all the privacy information would be leaked during the communication. Besides, the security of the sever is essential in previous system: once the server is attacked, all the privacy information like history messages would be accessed by adversary. In the secure E2EE chat system, the server is just responsible for storing users' key bundles and basic information. The loss would not be inestimable when the server is down.

The final product implements the E2EE feature in the system. Users could have pairwise chat and group chat secretly. The system not only develops the functions that secure the communication content between users, but also implements other reliable functions such as multi-device system, message backup and pairwise chat verification. The upgraded chat system could satisfy almost all the requirements of users in security aspect. However, the defects of the system like not appending pre keys and unexpected issues after continuous device switching influence the robustness and performance. Generally, the final artifact of this project achieves the goals that set at the beginning and solve the problem that could not satisfy users' security requirements previously.

\subsection{evaluation}
In general, the solution that using the Signal Protocol to improve the security of the chat system satisfy the users' security requirements. There are several advantages and disadvantages of the final product could be evaluated. The evaluation could be discussed in aspects including functionality, security and stability.

In functionality aspect, the core parts of the Signal related codes are invoked from the libsignal-protocol-java lib. After having a deep understanding about the Signal Protocol and reading the source code in detail, the Signal related functions are implemented properly. The users could chat with others in pairwise and group chat way as before. The upgraded system does not effect the functionality basically. The only alteration is that the server does not maintain the history messages. The user is required to save the Signal related storages and chat data.

In security aspect, all the security features including forward security, future security and the deniability are implemented inside the Signal Protocol. Since the Signal Protocol is implemented in the system properly, the security of users' communication content is guaranteed. But the system does not consider other type of attacks. The users privacy could be accessed via SQL injection attack and the chat data would be leaked once the encrypted storages are cracked in the client. So the transmission security of users is improved but the security of the whole system still needs evolution.

In stability aspect, the user could have continuous using experience because of the Signal storages and chat data storages. The system could work in both asynchronous and synchronous environments. Users could switch devices back and forward without message lost. The whole system works as expected in the case of a small number of users. But due to the previous structure of the project, the system may occur some issues in the case of a large number of users or in the case of continuous device switching. So the stability of the system is the weakest part, the final artifact is not suggested for business use. \nocite{GroupChat2} \nocite{Remain1} \nocite{Remain2} \nocite{Remain3} \nocite{Remain4} \nocite{Remain5}

\clearpage