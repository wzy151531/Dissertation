\section{Introduction}
\subsection{Overview}
The aim of the work described in this Report is to improve the security of a chat system based on Signal Protocol which people can be not worried about their privacy leakage.
\subsection{Problem statement}
Under the rapid development of the information age, the encryption protocol also have a new revolution. At the beginning, people transmit the messages in plain text which is a original way and information can be obtained easily. So symmetric encryptions born in this situation. The call parties need to negotiate a common cipher key to encrypt the messages during communication. Although the symmetric encryption is quite difficult to crack. But the benefits of cracking are huge. So maybe the hacker will monitor on several days' packages and spend some times to crack the encrypted messages to get the cipher key. And then the he can decrypt any other messages before or after without the knowledge of both parties. Besides, the common cipher key's exchange is also a problem, call parties need to negotiate it via a safe way such as face to face or the cipher key can be obtained by malicious people.

To solve the symmetric encryption's key exchange problem, asymmetric encryption is developed. The server and client will use some asymmetric algorithms to obtain the future common symmetric cipher key. The progress is the security of transmission is improved, the hacker cannot get the common symmetric cipher key easily while two parties are exchanging it. However, the problem mentioned before is still not solved: once the hacker cracks one of the packages, all the packages are exposed to the hacker which will cause unpredictable losses. Besides, the credit of server is also a problem. Usually the call parties are client and server, server decrypt encrypted messages from clients and then forward them to the corresponding users or store them in database. Which means, some unethical companies may sell users' chat history or the hackers can focus on attacking the database to get the information, both of them can have serious consequences.

In this situation, the public needs a new protocol to protect their privacy. Whisper is a good inspiration to develop a new protocol. Only the two parties of communication know what they are talking about. The server are only responsible for forwarding users' messages but not decrypting them. Also, the server does not store the decrypted even encrypted messages in database to make sure the security of users' privacy is guaranteed by users themselves. That is the original purpose of Signal Protocol. It not only combines the symmetric and asymmetric encryption, but also adds a new algorithm to provide forward and future security: Double Ratchet.

\subsection{Project aim}
This project is aim to implement an E2EE chat system using Signal Protocol which can improve the communication security of it. The project does not only develop an E2EE chat program, it focuses on a whole E2EE chat system including server, client, database sides. 

\subsection{Project structure}
The existing chat system is developed by jdk11 and gradle6. It can be divided into three parts: client, server and database. The client is developed using javafx11, it uses MVC architecture to link the data and view. In transmission aspect, it uses socket programming to communicate with server. The communication package has a specific format: each package has a type, both client and server will handle the package corresponding to each type.

On server side, it connects to the bham's database via bham's SSH service.
The database uses PostgresSQL database which served by bham.

The project's main work is on client side where the Signal Protocol encrypt the chat messages of user. The client implements several functions such as pairwise chat, group chat, chat history storage and history messages backup etc. On server side, most of the work is about how to handle the packages from client. For example, server needs to response to users' key bundle requests and forwards the packages to the right users. In database, it simplify the structure of tables, because there is no need to store the chat data in database. But some creation and modification are needed to achieve a Signal compatible system like create a new table to store users' key bundles, add deviceId field in users table to implement a multi-device system.

\subsection{Outline of each section}
In the next set of sections, the report introduce the further related knowledges and the specific implementation and the final result of the system.

In chapter 3, the report introduce some further background materials that reader needs to know to understand the solution. It focuses on how Signal Protocol works and the context of related existing work.

In chapter 4, the report describes the user specification of system, main design of solution and concrete implementation of upgrading the chat system using Signal Protocol including client, server, database sides. It also shows how the project and the writing of a substantial piece of software is managed. Finally, the report presents the whole implemented system's usage result to discuss its success and robustness.

In chapter 5, the report summarise the achievements and the deficiencies of the project. The things that could or would have done are listed for further evaluation.

In chapter 6, the report give a brief statement of how the solution addresses the problem stated at the beginning and provide an evaluative statement based on the results.

In chapter 7, the report explains the file structure of the code and introduces how to run the project code in detail. The proposal and schedules are also presented in this chapter.

\clearpage